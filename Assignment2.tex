%\documentclass[11pt,reqno]{amsart}
\documentclass[11pt,reqno]{article}
\usepackage[margin=.8in, paperwidth=8.5in, paperheight=11in]{geometry}
%\usepackage{geometry}                % See geometry.pdf to learn the layout options. There are lots.
%\geometry{letterpaper}                   % ... or a4paper or a5paper or ... 
%\geometry{landscape}                % Activate for for rotated page geometry
%\usepackage[parfill]{parskip}    % Activate to begin paragraphs with an empty line rather than an indent7
\usepackage{graphicx}
\usepackage{pstricks}
\usepackage{amssymb}
\usepackage{epstopdf}
\usepackage{amsmath}
\usepackage{subfigure}
\usepackage{caption}
\pagestyle{plain}
%\renewcommand{\topfraction}{0.3}
%\renewcommand{\bottomfraction}{0.8}
%\renewcommand{\textfraction}{0.07}
\DeclareGraphicsRule{.tif}{png}{.png}{`convert #1 `dirname #1`/`basename #1 .tif`.png}

\title{Probability and Random Variables: \\ Assignment 1}
\author{Andrew Rickert}
\date{Started: January ??, 2011 \\ \hspace{1pt} Ended: January ??,  2011}                                           % Activate to display a given date or no date

\begin{document}
\maketitle

\noindent \framebox[1.1\width]{\textbf{Part A}} \par

% Page 1
\begin{flushleft} 
\textbf{Class 18.440} - Chapter 2 Problem 27\\
\rule{500pt}{1pt}\\
\end{flushleft} 


\vspace{15pt}
\begin{flushleft} 
\textbf{Class 18.440} - Chapter 2 Problem 34\\
\rule{500pt}{1pt}\\
\end{flushleft} 


\vspace{15pt}
\begin{flushleft} 
\textbf{Class 18.440} - Chapter 2 Problem 49\\
\rule{500pt}{1pt}\\
\end{flushleft} 

\vspace{15pt}
\begin{flushleft} 
\textbf{Class 18.440} - Chapter 2 Problem 53\\
\rule{500pt}{1pt}\\
\end{flushleft} 

\vspace{15pt}
\begin{flushleft} 
\textbf{Class 18.440} - Chapter 2 Problem 54\\
\rule{500pt}{1pt}\\
\end{flushleft} 
 
\vspace{15pt}
\begin{flushleft} 
\textbf{Class 18.440} - Chapter 2 Theoretical Exercise 6\\
\rule{500pt}{1pt}\\
\end{flushleft} 


\vspace{15pt}
\begin{flushleft} 
\textbf{Class 18.440} - Chapter 2 Theoretical Exercise 7\\
\rule{500pt}{1pt}\\
\end{flushleft} 


\vspace{15pt}
\begin{flushleft} 
\textbf{Class 18.440} - Chapter 2 Theoretical Exercise 18\\
\rule{500pt}{1pt}\\
\end{flushleft} 


\vspace{15pt}
\begin{flushleft} 
\textbf{Class 18.440} - Chapter 2 Theoretical Exercise 20\\
\rule{500pt}{1pt}\\
\end{flushleft} 


\vspace{15pt}
\begin{flushleft} 
\textbf{Class 18.440} - Chapter 1 Self Test Problem/Exercise 1\\
\rule{500pt}{1pt}\\
\end{flushleft} 

\vspace{15pt}
\begin{flushleft} 
\textbf{Class 18.440} - Chapter 1 Self Test Problem/Exercise 14\\
\rule{500pt}{1pt}\\
\end{flushleft} 


\noindent \framebox[1.1\width]{\textbf{Part B}} \par

(Just for fun � not to hand in.) The following is a popular and rather instructive puzzle. A standard deck of 52 cards (26 red and 26 black) is shuffled so that all orderings are equally likely. We then play the following game: I begin turning the cards over one at a time so that you can see them. At some point (before I have turned over all 52 cards) you say �I�m ready!� At this point I turn over the next card and if the card is red, you receive one dollar; otherwise you receive nothing. You would like to design a strategy to maximize the probability that you will receive the dollar. How should you decide when to say �I�m ready�?

\end{document}  
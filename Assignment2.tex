%\documentclass[11pt,reqno]{amsart}
\documentclass[11pt,reqno]{article}
\usepackage[margin=.8in, paperwidth=8.5in, paperheight=11in]{geometry}
%\usepackage{geometry}                % See geometry.pdf to learn the layout options. There are lots.
%\geometry{letterpaper}                   % ... or a4paper or a5paper or ... 
%\geometry{landscape}                % Activate for for rotated page geometry
%\usepackage[parfill]{parskip}    % Activate to begin paragraphs with an empty line rather than an indent7
\usepackage{graphicx}
\usepackage{pstricks}
\usepackage{amssymb}
\usepackage{epstopdf}
\usepackage{amsmath}
\usepackage{subfigure}
\usepackage{caption}
\pagestyle{plain}
%\renewcommand{\topfraction}{0.3}
%\renewcommand{\bottomfraction}{0.8}
%\renewcommand{\textfraction}{0.07}
\DeclareGraphicsRule{.tif}{png}{.png}{`convert #1 `dirname #1`/`basename #1 .tif`.png}

\title{Probability and Random Variables: \\ Assignment 2}
\author{Andrew Rickert}
\date{Started: February 13, 2012 \\ \hspace{1pt} Ended: February 29,  2012}                                           % Activate to display a given date or no date

\begin{document}
\maketitle

\noindent \framebox[1.1\width]{\textbf{Part A}} \par

% Page 1
\begin{flushleft} 
\textbf{Class 18.440} - Chapter 2 Problem 27\\
\rule{500pt}{1pt}\\
\end{flushleft} 

An urn contains 3 red and 7 black balls. Players $A$ and $B$ withdraw balls from the urn consecutively until a red ball is selected. Find the probability that $A$ selects the red ball. ($A$ draws the first ball, then $B$, and so on. There is no replacement of the balls drawn.)\\

We need to find the total number of outcomes where player $A$ wins and divide this by the total number of possible outcomes. \\
Every win for Player $A$ will be a string black balls selected ending with the selection of a red ball. Since there are only 7 black balls there can only be 4 chances. After the 4th chance all the black balls have been selected which forces player $B$ to select a red. The following lists all the possible outcomes by color.\\

\begin{tabular}{l c }
Outcome & Winner \\ \hline
$(r)$ \quad & $A$ \\
$(b,r)$ \quad & $B$  \\
$(b,b,r)$ \quad & $A$ \\
$(b,b,b,r)$ \quad & $B$  \\
$(b,b,b,b,r)$ \quad & $A$  \\
$(b,b,b,b,b,b,r)$ \quad & $B$ \\
$(b,b,b,b,b,b,b,r)$ \quad & $A$ \\
$(b,b,b,b,b,b,b,b,r)$ \quad & $B$ \\ \\
\end{tabular}

\noindent There are four events where player $A$ wins:
\begin{align*} 
E_1 & = (r) \\ 
E_2 & = (b,b,r) \\ 
E_3 & = (b,b,b,b,r) \\ 
E_4 & = (b,b,b,b,b,b,r)
\end{align*}
Because each of these events represents an ordered selection from the urn we have $E_i \cap E_j = \emptyset $ for $i,j \in \{1,2,3,4\}$. This shows that:
\[ P(A \; \text{wins}) = P(E_1 \cup E_2 \cup E_3 \cup E_4) = P(E_1) + P(E_2) + P(E_3) + P(E_4) \]
So, we must find the probabilities for each the events. 

For $E_1$ there are 3 possibilities for choosing a red ball and 10 possibilities total so $P(E_1) = \frac{3}{10}$. For $E_2$ we repeat this process. There are 7 choices for the first black ball then 6 choices for the second black ball. Finally there are 3 choices for the final red ball. So there are $7 \cdot 6 \cdot 3$ outcomes that result in the event $(b,b,r)$. Since there are 10, choices for the first ball, 9 choices for the second ball and 8 choices for the third ball we have $P(E_2) = \frac{7 \cdot 6 \cdot 3}{10 \cdot 9 \cdot 8}$. We may proceed similarly for the last two events to get

\noindent There are four events where player $A$ wins:
\begin{align*} 
P(E_1) & = \frac{3}{10} \\ 
P(E_2) & = \frac{7 \cdot 6 \cdot 3}{10 \cdot 9 \cdot 8} \\ 
P(E_3) & = \frac{7 \cdot 6 \cdot 5 \cdot 4 \cdot 3}{10 \cdot 9 \cdot 8 \cdot 7 \cdot 6} = \frac{ 5 \cdot 4 \cdot 3}{10 \cdot 9 \cdot 8} \\ 
P(E_4) & = \frac{7 \cdot 6 \cdot 5 \cdot 4 \cdot 3 \cdot 2 \cdot 3}{10 \cdot 9 \cdot 8 \cdot 7 \cdot 6 \cdot 5 \cdot 4} =   \frac{3 \cdot 2 \cdot 3}{10 \cdot 9 \cdot 8}\\ \\
\implies & P(E_1) + P(E_1) + P(E_1) + P(E_1) = \frac{3}{10} + \frac{7 \cdot 6 \cdot 3}{10 \cdot 9 \cdot 8} + \frac{ 5 \cdot 4 \cdot 3}{10 \cdot 9 \cdot 8} + \frac{3 \cdot 2 \cdot 3}{10 \cdot 9 \cdot 8}\\
\implies &P(A \; \text{wins}) =  3\cdot \frac{9 \cdot 8 + 7 \cdot 6 + 5 \cdot 4 + 3 \cdot 2 }{10 \cdot 9 \cdot 8} = .58333
\end{align*}


\vspace{15pt}
\begin{flushleft} 
\textbf{Class 18.440} - Chapter 2 Problem 34\\
\rule{500pt}{1pt}\\
\end{flushleft} 

The second Earl of Yarborough is reported to have bet at odds of 1000 to 1 that a bridge hand of 13 cards would contain at least one card that is ten or higher. (By \emph{ten or higher} we mean that a card is either a ten, a jack, a queen, a king, or an ace.) Nowadays, we call a hand that has no cards higher than 9 a \emph{Yarborough}. What is the probability that a randomly selected bridge hand is a Yarborough?\\

If we consider each individual hand being drawn as equally likely then the probability will be the number of Yarboroughs divided by the total number of possible hands. \\
There are $\binom{52}{13}$ possible 13 card hands. Since there are 4 each of the 10, jack, queen, king, ace, there are 52 - 20 = 32 cards from which a Yarborough can be drawn. So the probability of a Yarborough is 
\[ P(\text{Yarborough}) = \frac{\binom{32}{13}}{\binom{52}{13}} \backsimeq 0.000547 \]

\noindent So the chances are 1828:1 that a Yarborough will be drawn. A good bet for the Earl.

\vspace{15pt}
\begin{flushleft} 
\textbf{Class 18.440} - Chapter 2 Problem 49\\
\rule{500pt}{1pt}\\
\end{flushleft} 

A group of 6 men and 6 women is randomly divided into 2 groups of size 6 each. What is the probability that both groups will have the same number of men?\\

Since there are 6 men the only way they can be divided equally into two groups is to have 3 men in each group. Assuming each group is equally likely to be chosen the probability will be the number of ways three men can be placed in each group divided by the total number of possible groups.\\
The number of possible groups will be the number of ways we can choose 6 people for the first group, $\binom{12}{6}$, then the number of ways we can choose the remaining 6 people for the second group, $\binom{6}{6}$. Since the first group and the second group are not distinguishable the total number of groups is:
\[ \text{Total} = \frac{1}{2!}\binom{12}{6}\binom{6}{6} \]

For the first group of three men there are $\binom{6}{3}$ ways to choose the men and $\binom{6}{3}$ ways to choose the women. So we have
\[ \text{Group 1 Total} = \binom{6}{3}\binom{6}{3} \]

Group 2 must choose from among the remaining 3 men and 3 women, so
\[ \text{Group 2 Total} = \binom{3}{3} \binom{3}{3} \]

Since the order in which the groups are chosen is irrelevant the total number of ways to chose the groups with 3 men is 

\[ \text{3 Men Group total} = \frac{1}{2!} \left( \binom{6}{3}\binom{6}{3} \cdot \binom{3}{3}\binom{3}{3}   \right) =   \binom{6}{3}\binom{6}{3} \]

So the probability is 
\[ P(\text{3 men groups})= \frac{ \frac{1}{2!}\binom{6}{3}\binom{6}{3}}{\frac{1}{2!}\binom{12}{6}} = \frac{\binom{6}{3}^2}{\binom{12}{6}} = \frac{100}{231} \backsimeq 0.4329 \]

\vspace{15pt}
\begin{flushleft} 
\textbf{Class 18.440} - Chapter 2 Problem 53\\
\rule{500pt}{1pt}\\
\end{flushleft} 

\noindent If 4 married couples are arranged in a row, find the probability that no husband sits next to his wife.\\

We let the event $E_i$ be the event that couple $i$ sits together. The probability that no husband sits next to his wife is the same as the probability that it doesn't happen that at least one husband sits with his wife. That is
\[ P(\text{No husband sits with wife}) = P((E_1 \cup E_2 \cup E_3 \cup E_4)^c) = 1 - P(E_1 \cup E_2 \cup E_3 \cup E_4) \]

\noindent Our goal is then to calculate $P(E_1 \cup E_2 \cup E_3 \cup E_4)$. We will use the inclusion-exclusion identity to finish the calculation so we will need to calculate the probabilities of the events one at time, two at time, etc. 

The total number of seating arrangements is $8!$ by the basic counting principle. To calculate the probabilities of the events, one at a time, two at a time, etc. we need to find the number of arrangements that result in these events.\\
When one couple sits together they take up two seats. The remaining 6 seats can be distributed to the left and/or right of the couple. If we let $x_1$ be the number seats on the left and $x_2$ be the number of seats on the right then the equation that determines the distribution of the empty seats is
\[ x_1 + x_2 = 6\]
The number of solutions to this equation where $x_1,x_2 \ge 0$ is $\binom{n+r -1}{r-1}$. Here $n = 6$ and $r = 2$ so there are $\binom{7}{1}$ distributions. For each distribution there are $6!$ ways to seat the remaining 6 people and $2!$ ways to seat the couple so we get
\[ P(E_i) = \frac{\binom{7}{1} 6! 2! }{8!} = \frac{7 \cdot 6! 2!}{8!} = \frac{7! 2!}{8!}\]

Now we seat two couples. There are most 4 seats to the left of couple 1, 4 seats between couples 1 and 2, and 4 seat to the right of couple 2. This leads to the following equation for the distribution of the empty seats.
\[ x_1 + x_2 + x_3 = 4 \]
The number of distributions is then $\binom{4+3-1}{3-1} = \binom{6}{2}$. For each distribution of seats there are $4!$ ways to arrange the remaining 4 people. For each arrangement there are also $2!$ arrangements of the two couples and $2!2!$ ways to arrange the people in each couple. This gives the probability as
\[ P(E_i E_j) = \frac{\binom{6}{2} 4! 2! 2! 2!}{8!} = \frac{6!2!2!}{8!}\]
Similarly for three couples there are 2 empty spaces which can be left of couple 1, right of couple 3 or in between the couples. This leads to the equation
\[ x_1 + x_2 + x_3 + x_4 = 2 \]
There are $\binom{2 + 4 - 1}{4 - 1} = \binom{5}{3}$ possible distributions of the 4 spaces. There are 2! ways to distribute the two people not in couples. There are also $3!$ arrangements of the couples for any given distribution of spaces and $2!2!2!$ ways to arrange the people in each couple. This gives the following probability
\[ P(E_i E_j E_k) = \frac{\binom{5}{3} 2! 3! 2! 2! 2!}{8!} = \frac{5!2!2!2!}{8!} \]

For $P(E_1 E_2 E_3 E_4)$ there is only 1 way to distribute the zero spaces. There are $4!$ ways to arrange the couples and $2!2!2!2!$ ways to arrange the people in each couple so
\[ P(E_1 E_2 E_3 E_4) = \frac{4! 2!2!2!2!}{8!} \]

Note, that the probabilities are the same for each set of intersected events. For example \\$P(E_1) = P(E_2) = \ldots $ or $P(E_1 E_2 E_3) = P(E_2 E_3 E_4) = \ldots $\\
By the inclusion exclusion principle we have
\begin{eqnarray*}
P(E_1 E_2 E_3 E_4) &=& \sum_{k=1}^4 P(E_i) - \sum_{i_1 < i_2} P(E_{i_1} E_{i_2}) + \sum_{i_1 < i_2 < i_3} P(E_{i_1} E_{i_2} E_{i_3}) - \sum_{i_1 < i_2 < i_3 < i_4} P(E_{i_1} E_{i_2} E_{i_3} E_{i_4})\\
&=& \binom{4}{1} P(E_1) - \binom{4}{2} P(E_1 E_2) + \binom{4}{3} P(E_1 E_2 E_3) - \binom{4}{4} P(E_1 E_2 E_3 E_4) \\
&=& 4 \cdot \frac{7! 2!}{8!} - 6 \cdot \frac{6!2!2!}{8!} + 4 \cdot \frac{5!2!2!2!}{8!} - \frac{4! 2!2!2!2!}{8!} \\
&=& 1 - \frac{3}{7} + \frac{2}{21} - \frac{1}{105} = \frac{23}{35}
\end{eqnarray*}
This gives the answer:\\
\[P(\text{No husband sits with wife}) = 1 - P(E_1 \cup E_2 \cup E_3 \cup E_4) = \frac{12}{35} \backsimeq 0.34286 \]
\newpage

\vspace{15pt}
\begin{flushleft} 
\textbf{Class 18.440} - Chapter 2 Problem 54\\
\rule{500pt}{1pt}\\
\end{flushleft} 
 Compute the probability that a bridge hand is void in at least one suit. Note that the answer is not
\[ \frac{\binom{4}{1} \binom{39}{13}}{\binom{52}{13}} \]
(Why not?)
Hint: Use Proposition 4.4 \\
 
The expression $ \frac{\binom{4}{1} \binom{39}{13}}{\binom{52}{13}} $ only calculates the probability of drawing a bridge hand miss exactly one suit. We need to find the probability it is missing one suit, two suit or three suits.\\
We let $E_1,E_2,E_3,E_4$ be the events that a bridge hand is missing clubs,spades,hearts or diamonds respectively.\\
The probability that one or more of the suits is missing is therefore $ P(E_1 \cup E_2 \cup E_3 \cup E_4)$. We will use the inclusion exclusion theorem to find this probability but in order to do this we need the probability of the events taken one at a time, two at time, etc.\\
If there is one suit missing then there will be only 52-13 = 39 cards to chose a 13 card hand from. This gives the probability for events taken one at a time.
\[ P(E_i) = \frac{\binom{39}{13}}{\binom{52}{13}}\]
If there two suits missing then there will be only 52-26 = 26 cards to chose a 13 card hand from. This gives the probability for events taken one at a time.
\[ P(E_i E_j) = \frac{\binom{26}{13}}{\binom{52}{13}}\]
 If there are three suits missing there there are only 13 cards from which to make a 13 card hand. That is, there is one such hand to made. The probability is
\[ P(E_i E_j E_k) = \frac{\binom{13}{13}}{\binom{52}{13}} = \frac{1}{\binom{52}{13}}\] 
 There is no 13 card hand missing all of the suits since this would eliminate all of the cards so
 \[ P(E_1 E_2 E_3 E_4) = 0 \]
 
We may now apply the inclusion exclusion theorem to get the probability for drawing a hand missing at least one suit
 \begin{eqnarray*}
P(E_1 E_2 E_3 E_4) &=& \sum_{k=1}^4 P(E_i) - \sum_{i_1 < i_2} P(E_{i_1} E_{i_2}) + \sum_{i_1 < i_2 < i_3} P(E_{i_1} E_{i_2} E_{i_3}) - \sum_{i_1 < i_2 < i_3 < i_4} P(E_{i_1} E_{i_2} E_{i_3} E_{i_4})\\
&=& \binom{4}{1} P(E_1) - \binom{4}{2} P(E_1 E_2) + \binom{4}{3} P(E_1 E_2 E_3) - \binom{4}{4} P(E_1 E_2 E_3 E_4) \\
&=& 4 \frac{\binom{39}{13}}{\binom{52}{13}} - 6 \frac{\binom{26}{13}}{\binom{52}{13}} + 4  \frac{1}{\binom{52}{13}} \backsimeq 0.0511
\end{eqnarray*}
 
\vspace{15pt}
\begin{flushleft} 
\textbf{Class 18.440} - Chapter 2 Theoretical Exercise 6\\
\rule{500pt}{1pt}\\
\end{flushleft} 

Let $E$, $F$, and $G$ be three events. Find expressions for the events so that, of $E$, $F$, and $G$,\\
(a) only $E$ occurs;\\
(b) both $E$ and $G$, but not $F$, occur;\\
(c) at least one of the events occurs;\\
(d) at least two of the events occur;\\
(e) all three events occur;\\
(f) none of the events occurs;\\
(g) at most one of the events occurs;\\
(h) at most two of the events occur;\\
(i) exactly two of the events occur;\\
(j) at most three of the events occur.\\

\noindent (a) $E F^c G^c$\\
\noindent (b) $E G F^c$\\
\noindent (c) $E \cup F \cup G$\\
\noindent (d) $E F \cup E G \cup F G$\\
\noindent (e) $E F G$\\
\noindent (f) $(E \cup F \cup G)^c$\\
\noindent (g) $E F^c G^c \cup E^c F G^c \cup E^c F^c G$\\
\noindent (h) $(EFG)^c$\\%$E F G^c \cup E G F^c \cup E^c F G$\\
\noindent (i) $E F G^c \cup E G F^c \cup F G E^c$\\
\noindent (j) $E \cup F \cup G$\\


\vspace{15pt}
\begin{flushleft} 
\textbf{Class 18.440} - Chapter 2 Theoretical Exercise 7\\
\rule{500pt}{1pt}\\
\end{flushleft} 

Find the simplest expression for the following events:\\
(a) $(E \cup F)(E \cup F^c)$\\
(b) $(E \cup F)(E^c \cup F)(E \cup F^c)$\\
(c) $(E \cup F)(F \cup G)$\\

\noindent(a)
\begin{align*}
\hspace{16pt} (E \cup F)(E \cup F^c) &= E \cup (F F^c) \quad \text{By the distributive law}\\
  				  &= E \cup \emptyset \quad \text{By the definition of the complement and intersection}\\
				  &= E
\end{align*}

\noindent(b)
\begin{align*}
\qquad (E \cup F)(E^c \cup F) (E \cup F^c) &= (E E^c \cup F)(E \cup F^c)\quad \text{By the distributive law}\\
  				  &= (\emptyset \cup F)(E \cup F^c) \quad \text{By the definition of the complement and intersection}\\
				  &= F(E \cup F^c)\\
				  &= F E \cup F F^c \quad \text{By the distributive law}\\
				  &= F E \cup \emptyset \quad \text{By the definition of the complement and intersection}\\
				  &= FE
\end{align*}

\noindent(c)
\begin{align*}
\hspace{-68pt}(E \cup F)(F \cup G) &= (E \cup F)(G \cup F) \quad \text{By the commutative law}\\
  			        		   &= EG \cup F  \quad \text{By the distributive law}\\
\end{align*}

\vspace{15pt}
\begin{flushleft} 
\textbf{Class 18.440} - Chapter 2 Theoretical Exercise 18\\
\rule{500pt}{1pt}\\
\end{flushleft} 

Let $f_n$ denote the number of ways of tossing a coin $n$ times such that successive heads never appear. Argue that
\[ f_n = f_{n-1} + f_{n-2} \quad n \ge 2, \; \text{where} \; f_0 \equiv 1, f_1 \equiv 2 \]

Hint: How many outcomes are there that start with a head, and how many start with a tail? If $P_n$ denotes the probability that successive heads never appear when a coin is tossed $n$ times, find $P_n$ (in terms of $f_n$) when all possible outcomes of the $n$ tosses are assumed equally likely. Compute $P_{10}$.\\

Consider a sequence of $n$ coin flips. This sequence either begins with a head or begins with a tail. This means that we can break the total number of outcomes, $f_n$, into the number of outcomes that start with a head, $A_n$ and the number of outcomes that start with a tails, $B_n$.
\[ f_n = A_n + B_n \]

For $B_n$ we notice that since the first element in all of the sequences is a tail the number of outcomes for $B_n$ is the number of outcomes for sequences starting after the first tails. There are $n-1$ coin flips, and since there are no restrictions on these sequences different from the sequence of $n$ coin flips we find that $B_n = f_{n-1}$.
For a sequence to be among those in $A_n$ there is a further restriction. Because all the sequences in $A_n$ start with a head the second element must be a tails. This means the first two values are fixed and the remaining values are not restricted. By the logic of the previous paragraph this means that $A_n = f_{n-2}$. So we have shown that $f_n = f_{n-1} + f_{n-2}$.\\
\indent To find the probability of not having successive heads in a flipping a coin $n$ times we need to divide the number of ways this can happen, $f_n$ by the number of possible coin flip sequences.
Since there are two possibilities for each coin flip there are $2^n$ possible sequences of $n$ coin flips by the basic counting principle. If $E_n$ is the event of no successive heads in $n$ coin flips then the probability is:
\[ P_n = P(E_n) = \frac{f_n}{2^n} \]

Using the proven relationship for $f_n$ we can derive $f_{10}$ from $\{f_0, f_1,f_2,f_3,f_4,f_5,f_6,f_7,f_8,f_9,f_10 \}$ by adding the successive values to get $\{ 1,2,3,5,8,13,21,34,55,89,144\}$, so $f_{10} = 144$ and the probability is 
\[ P_{10} = \frac{144}{2^{10}} = \frac{144}{1024} \backsimeq .1406\]
\newpage

\vspace{15pt}
\begin{flushleft} 
\textbf{Class 18.440} - Chapter 2 Theoretical Exercise 20\\
\rule{500pt}{1pt}\\
\end{flushleft} 

Consider an experiment whose sample space consists of a countably infinite number of points. Show that not all points can be equally likely. Can all points have a positive probability of occurring? \\ 

To show that it is not possible that an countably infinite set of points can be equally likely we let $i \in \mathbb{N}$ be an index for the elements in the sample space. This is possible since the sample space $S$ is countable infinite. Now let the $s_i$ be $i$th element in $S$  $P(\{ s_i \}) = b \ge 0$. So $b$ represents the probability for each point in the sample space. \\
If we now calculate the probability:
\[ P(S) = P(\cup_{i=1}^\infty \{ s_i \})  = \sum_{i = 1}^\infty P(\{s_i\}) =  \sum_{i = 1}^\infty b \]

The probability axioms say that it must be true that $P(S) = 1$, if $b = 0$ then the above calculation shows that $P(S) = 0$. If $b \neq 0$ then the above calculation shows that $P(S) = \infty$. In all cases the assumption of uniform probability for a countably infinity sample space violates one of the axioms for the probability function so the assumption must be invalid.\\
\indent A countably infinite sample space \emph{can} have positive probability for all points. If we have as our sample space $S = \{ s_1, s_2, s_3, \ldots \}$ and have $P(\{ s_i\}) = \frac{1}{2^i}$ then we have 
\[ P(S) = P(\cup_{i=1}^\infty \{ s_i \})  = \sum_{i = 1}^\infty P(\{s_i\}) =  \sum_{i = 1}^\infty \frac{1}{2^i}  = 1\]
It is also true that $\frac{1}{2^i} > 0$ for all $i \in \mathbb{N}$ so the probabilities are all positive.


\vspace{15pt}
\begin{flushleft} 
\textbf{Class 18.440} - Chapter 2 Self Test Problem/Exercise 1\\
\rule{500pt}{1pt}\\
\end{flushleft} 

A cafeteria offers a three-course meal consisting of an entree, a starch, and a dessert. The possible choices are given in the following table:

\begin{center}
    \begin{tabular}{ | l | l | }
    \hline
    Course & Choices  \\ \hline
    Entree & Chicken or roast beef   \\ %\hline
    Starch & Pasta or rice or potatoes  \\ %\hline
    Dessert & Ice cream or Jello or apple pie or a peach \\ \hline
    \end{tabular}
\end{center}

\noindent A person is to choose one course from each category.\\
(a) How many outcomes are in the sample space?\\
(b) Let A be the event that ice cream is chosen.\\
How many outcomes are in A?\\
(c) Let B be the event that chicken is chosen.\\
How many outcomes are in B?\\
(d) List all the outcomes in the event AB.\\
(e) Let C be the event that rice is chosen. How many outcomes are in C?\\
(f) List all the outcomes in the event ABC\\

\noindent Let $N(E)$ be the number of outcomes in the event $E$.\\
\noindent (a) 2 Entrees, 3 starches, 4 deserts. Basic counting principle says total number of outcomes is $2 \cdot 3 \cdot 4 = 24$. 
\noindent (b) There are 2 entrees to select then 3 starches to select so $N(A) = 2 \cdot 3 = 6$ outcomes.\\
\noindent (c) There are 3 starches and 4 desserts to select so $N(B) = 3 \cdot 4 = 12$ outcomes.\\
\noindent (d) Since is $A$ is the event that ice cream is chosen and $B$ is the event that chicken is chose $A B$ is the event that ice cream and chicken are chosen. The Starch is the only free variable so the outcomes are \\
\[\{ (\text{Chicken,pasta,Ice Cream}),(\text{Chicken,rice,Ice Cream}),(\text{Chicken,potatoes,Ice Cream}) \}\]
(e) There are 2 entrees and 4 desserts to chose from so $N(C) = 2 \cdot 4 = 8$ outcomes.\\
(f) The event $A B C$ is the event that Chicken, rice, and ice cream are chosen. So, the only outcome is \[ \{ (\text{Chicken, Rice, Ice Cream})\} \]

\vspace{15pt}
\begin{flushleft} 
\textbf{Class 18.440} - Chapter 2 Self Test Problem/Exercise 14\\
\rule{500pt}{1pt}\\
\end{flushleft} 

Prove Boole�s inequality:
\begin{equation*} 
P\left( \cup_{i=1}^\infty A_i \right) \le \sum_{i=1}^\infty P(A_i)
\end{equation*}

If we let $A_n = \cup_{i = 1}^n A_i$ then it is clear that $A_n$ is an increasing sequence of events since further unions only add to the number of elements in a set. If we have an increasing or decreasing family of sets then by theorem $6.1$ in the book we have 
\[ P(\lim _{n \to \infty} A_n) = \lim_{n \to \infty} P(A_n) \]

\noindent By a consequence of the inclusion - exclusion identity we have equation (4.1)
\[ P(\cup_{i = 1}^n A_i) \le \sum_{i = 1}^n P(A_i) \]

\noindent With theorem 6.1 and equation (4.1) we can calculate
\begin{eqnarray*}
P\left( \cup_{i=1}^\infty A_i \right) &=& P(\lim_{n \to \infty} \cup_{i = 1}^n A_i) \\
						   &=& P(\lim_{n \to \infty} A_n) \\
						   &=& \lim_{n \to \infty} P(A_n) \quad \text{by theorem 6.1}\\						    
						   &=& \lim_{n \to \infty} P(\cup_{i = 1}^n A_i)\\
						   &\le& \lim_{n \to \infty} \sum_{i = 1}^n P(A_i) \quad \text{by equation 4.1}\\
						   &=& \sum_{i = 1}^\infty P(A_i)
\end{eqnarray*}
So we have shown $P\left( \cup_{i=1}^\infty A_i \right) \le \sum_{i=1}^\infty P(A_i)$ which is the result. \\ 
\newpage

\noindent \framebox[1.1\width]{\textbf{Part B}} \par

(Just for fun - not to hand in.) The following is a popular and rather instructive puzzle. A standard deck of 52 cards (26 red and 26 black) is shuffled so that all orderings are equally likely. We then play the following game: I begin turning the cards over one at a time so that you can see them. At some point (before I have turned over all 52 cards) you say �I�m ready!� At this point I turn over the next card and if the card is red, you receive one dollar; otherwise you receive nothing. You would like to design a strategy to maximize the probability that you will receive the dollar. How should you decide when to say �I�m ready�?\\

If we have seen $i$ red cards and $j$ black cards then there are $52-i-j$ cards left. The number of patterns that begin with a red card are the number of ways of choosing positions for the remaining $26-i-1$ red cards from the total $52-i-j-1$ cards, that is $\binom{52-i-j-1}{26-i-1}$. The total number of patterns orderings of red or black cards is then $\binom{52-i-j}{26-i}$. The probability that the next card is red is therefore:
\[ P(\text{card}_{i+j+1} \; \text{is red}) = \frac{\binom{52-i-j-1}{26-i-1}}{\binom{52-i-j}{26-i}} = \frac{26-i}{52-i-j}\]


\end{document}  